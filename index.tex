\documentclass[a4paper,11pt]{article}

\htmlcss{style.css}
\htmltitle{MLCuddIDL}
\htmlpanel{0}
\setcounter{htmlautomenu}{1}
\setcounter{htmldepth}{1}

\usepackage{hyperlatex}
\usepackage{xspace}
%\usepackage{frames}

\newcommand{\ocaml}{\xlink{OCaml}{http://www.caml.org}\xspace}
\newcommand{\cudd}{\xlink{CUDD}{http://vlsi.colorado.edu/\~{}fabio/CUDD/cuddIntro.html}\xspace}
\newcommand{\cuddaux}{\xlink{CUDDAUX}{../cuddaux/index.html}\xspace}
\newcommand{\camlidl}{\xlink{CamlIDL}{http://caml.inria.fr/camlidl/}\xspace}
\newcommand{\camlmod}[1]{\xlink{#1}{html/#1.html}\xspace}

\title{MLCuddIDL}
\date{}
\author{}

\begin{document}

%\xmlattributes*{img}{align="left"}
%\xlink{\htmlimg{http://devel.inria.fr/logo_inria.png}{INRIA}}{http://www.inria.fr}
\xlink{Up}{../index.html}
\maketitle

\section{About}

MLCuddIDL is a C library offering an interface to the \cudd BDD
library for \ocaml version 3.00 or higher. The interface offers
access to most of the functions of the library, and in addition
implements some new functions. It is provided as a single module
\texttt{Cudd} containing submodules.

The interface is organized as follows:
\begin{itemize}
\item the module \camlmod{Man} implements functions common to all the
  other modules (initialization, variable ordering, ...) ;
\item the module \camlmod{Bdd} allows to manipulate ordinary BDDs (with
  complemented edge) ;
\item the module \camlmod{Add} allows to manipulate ADDs (Algebraic Decision
  Diagrams), the leaves of which are reals ;
\item the modules \camlmod{Mtbdd}, \camlmod{Mtbddc},
  \camlmod{User} and \camlmod{Mapleaf} allows to manipulate MTBDDs
  (Multi-Terminal Binary Decision Diagram), the leaves of which
  are a user-defined OCaml datatype; the user can define and
  register new operations, that will then take advantage of
  caching techniques .
\end{itemize}
I did not (yet) interfaced ZDDs, but I could do it quite quickly
is someone requests it.

The interface is clean for garbage collection (it makes use of custom
blocks). In addition, the garbage collectors of CUDD and OCaml are
synchronized: the OCaml garbage collector is requested to perform a
"full major" cycle before any CUDD garbage collection (in order to
allow the later to be more efficient). Serialization and
deserialization operations are not implemented. 

Feel free to ask me any feature you need and which is not implemented !

\section{License} 
LGPL

\section{Requirements}
\begin{itemize}
\item An ANSI C compiler (only gcc with ansi option has been really
  tested)
\item The \cudd BDD library
\item The \ocaml system (of course !)
\item The \camlidl stub code generator 
\end{itemize}

\section{Download}

\begin{itemize}
%\item \xlink{Tar-gzipped sources}{http://gforge.inria.fr/projects/mlxxxidl/}
\item \xlink{Subversion repository}{http://gforge.inria.fr/plugins/scmsvn/viewcvs.php/?root=mlxxxidl}
\end{itemize}

\section{Documentation}
\begin{itemize}
\item \xlink{Changes}{Changes}
\item \xlink{On-line}{html/index.html}
\item \xlink{PDF}{mlcuddidl.pdf}
\end{itemize}

\end{document}
